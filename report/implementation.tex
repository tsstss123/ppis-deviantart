\subsection{Toolkit introduction}
4 components, what should it do (online/offline), image2image, gal2gal, cat2cat, albert drawing

\subsection{Data collection}
\subsection{Feature extraction}
\subsection{Classification}
\subsection{Visualization}

% toolboxes used
\subsection{Data collection}
[Sander, Bart B.]
% xml format
% network

DeviantArt provides users, also called deviants, galleries of their images. 
When someone visits such a gallery a featured page will be shown. Furthermore
the user is provided with options to browse the gallery or visit a sub gallery
defined by the user. There are various types of members: normal members, premium 
members, banned members, staff members, etc. Premium members have extra benefits
like no ads, gallery customization, beta-test new site features and more.

DeviantArt does not provide a web api to download images. This makes it more 
difficult to download images. On top of that changes to the website can possible break
down the downloading application.

DeviantArt does provide rss, which allows us to download xml files containing 
information about the users galleries. RSS XML files are more easy to parse
than the html gallery pages.

For each image the full sized image and two different sized thumbnails are 
available. DeviantArt supports png, jpeg, bmp and gif image formats.

For the data collection of the galleries we followed the backend links to
the RSS XML files. Our \textit{scraper} is implemented in python and uses
built-in libraries to parse the XML files. For each image we stored general 
information like the category, deviantART link and filename and we downloaded
the full image and the thumbnails.

For the network information collection we parsed the HTML watchers pages
of the users, since no RSS XML files are provided by deviantART for this
information.

\subsection{Feature extraction}
When working with images, it is usually not possible to work with the raw image data (the pixel values). The reason for this is the high dimensionality of images, which can easily exist in a space of more than a million dimensions. By extracting features from images, they can be represented in a lower dimensional feature-space.  This feature extraction process has several advantages:
\begin{itemize}
\item The data becomes computationally easier to work with due to the smaller number of dimensions
\item By using the right features, the data becomes more suitable for generalization across images
\item Reducing the dimensionality makes it easier to visualize sets of images
\item Features can have an intuitive basis, which makes it easier for non-computer-scientists to analyze (sets of) images
\end{itemize}
\textit{Here something general about different kinds of image features}

% Computer vision is an important and maturing engineering science. It underpins an increasing variety of applications that require the acquisition, analysis, and interpretation of visual information.
In the extraction of image features, a distinction was made between low-level statistical features and higher level cognitive-based features....

\subsubsection{Statistical features}
As statistical features, many relatively simple low-level features were extracted from the images.
The first type of statistical features are colour-based features, which should capture the color-usage in the artwork. To capture

%\begin{itemize}
%\item Edge ratio: description
%\item Corner ratio: description
%\item etc...
%\end{itemize}

\subsubsection{Cognitively-inspired features}
One of the more recent trends in computer vision research in the pursuit of human-like capability is the coupling of cognition and vision into cognitive computer vision. 
The term cognitive computer vision has been introduced with the aim of achieving more robust, resilient, and adaptable computer vision systems by endowing them with a cognitive faculty: the ability to learn, adapt, weigh alternative solutions, and develop new strategies for analysis and interpretation.

In our project we focus on computational models of focal visual attention. Attention allows us to break down the problem of understanding a visual scene into a rapid series of computationally less demanding, localized visual analysis problems. 
"Visually salient" are those location in the visual wolrd that automatically attract attention.

\subsection{Classification}
[Quang]
% which classifiers
% how they work
% how they operate in the system
\subsection{Visualization}
[Nick]
% 3 views
