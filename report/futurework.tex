The toolkit proposed is offering the elementary components to explore part of dA's dataset and to perform some experiments on it. By extending its functionalities many more interesting information could be shown and more experiments could be conducted.

An interesting future investigation could focus on studying the core of the network structure with the aim of answering the following question: \textit{"can the core be representative for the whole network?"}
Even more, the network could be made to include more than only \textit{watcher-watching} connections, for example \textit{hierarchical taxonomy} connections. Moreover including all artworks  and their connections to their users and classes within the taxonomy would enrich the network.

An other attracting question concerning the network that could be answered in the future could be: \textit{"how the hierarchical artwork classification relates to sub-networks in the core of artists producing artwork in a particular class?"}.
DA users describe the existence of hubs, artists who have obtained popularity not by producing original art but as a  hub linking interesting artwork through favorites. Future research could try to to identify such types of artists based on their behavior in the network. These hubs could be found by relating popularity with the amount of links an artists provides to other persons artworks versus links to his original artwork.
Last, incorporating time in the user network would enable us to relate our network to the structure and evolution of the \textit{flickr} network as described in \cite{kumar2006structure} and  \cite{leskovec2008microscopic}.
%If we can identify different usage patterns of deviantART, for instance hubs which are artists who are not popular because of the artwork they themselves produce.
%But because of the other artists they connect to versus original artists who are popular because of their own work.

It would also be interesting to increase the number of extracted features, especially features that are based on the analysis of human perception.
The emotional impact of an artwork plays a significant role in the artistic creative process, and therefore the extraction of emotion-inspired features, such as color weight, color activity and color heat \cite{color_emotion1} could help the classifier to better distinguish between classes.
Moreover, the inclusion of texture would give an even deeper emotional descriptor~\cite{LucassenECCGIV2010}.

Even more, incorporating time into the toolkit would be a great addition, allowing experiments to investigate changes in art over time.

The visualization currently lacks advanced interactions to reduces clutter.
It would be nice to incorporate more advanced interactions like zooming and dynamic scaling.
Furthermore, the automated communication between the visualization and the other components of the toolkit could be done. This allows the user to dynamically expand the dataset by performing an action in the visualization application. 
Communication between the visualization and the classifier can be for instance be used to integrate classification inside the visualization application, allowing its usage without requiring technical knowledge.
%Use network better to guide our research, use network to pick dataset.
%recommender system
%doing it on a different dataset

%\item More network information - Using the Core network as a basis for a new dataset (ongoing), More links, not only watchers (hierachy)