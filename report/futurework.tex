%Use network better to guide our research, use network to pick dataset.
%recommender system
%doing it on a different dataset

%\begin{itemize}
%\subsubsection{more features - this header can later be removed}
%\item More features, Including emotional features (color and texture)
It could be interesting to increase the number of extracted features, especially the ones coming from anlysis of human perceptions.
For instance, the emotional impact of an artwork plays a significant role in the artistic creative process, and therefore the extraction of emotion-inspired features, such as color weight, color activity and color heat \cite{color_emotion1} could help the classifier to better distinquish betweent classes. Moreover, the inclusion of texture would give an even deeper emotional descriptor \cite{LucassenECCGIV2010}. 
%\item More network information - Using the Core network as a basis for a new dataset (ongoing), More links, not only watchers (hierachy)
%\subsubsection{more network- this header can later be removed}
We want to expand the network to include more than only "watcher-watching" connections. DeviantArt is a rich dataset from which we can also extract the "artist-artwork", "artwork-hierarchical class", "hierarchiecal taxonomy" -connections, as well as comments users make to artists and/or pictures. 
Since our current investigations have provided us with a managable set of core users for deviantArt, this group can guide us in dataset collection for future work. Relevant questions can be, how the hierarchical artwork classification relates to sub-networks in the core of artists producing artwork in a particular class. Another relevant question is if the network structure and artwork in the core is representative for the whole network. If we can identify different usage patterns of DeviantART, for instance 'hubs' which are artists who are not popular because of the artwork thei themselve produce. But because of the other artists they connect to versus original artists who are popular because of their own work.

 
%\item Incorporating time
%\item Using classifiers to make recommendations
%\end{itemize}