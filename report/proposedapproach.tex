% Here the formulas and description of what we used theoretically
%\documentclass{article}
%\begin{document}
\label{proposed}
\subsection{Networks}\label{proposed_networks}
Three network models of particular relevance for this project are the small-world model \cite{watts1998collective} from Watts and Strogatz and the Erd\"{o}s-R\'{e}nyi model for random networks \cite{erdos1960evolution} and regular ring latices. Small-world networks inhabit the space between the regular networks or lattices and random networks. Many practical networks have been shown to have a small-world topology, such as the nternet, the power grid amd neural networks,  for some more examples see \cite{albert2002statistical}. Small world-networks have a high cluster coefficient and low characteristic path length, means that  whereas random networks have a low characteristic path and low cluster coefficient and lattices have high cluster coefficients but long characteristic path length.

Coloquially explained in terms of my social network, a large(close to 1 ) cluster coefficient means that many of my friends know each other. A low characteristic path length means that I am connected to anyone on the world through small friends-of-friends chains.

Formally, let $G=(V, E)$ be a graph, where $V$ is a set of vertices, and $E$ a set of edges between vertices in $V$, here $e_{ij}$ denotes the edge connecting $i$ and $j$.  The \textit{out-neighborhood} of $v_i$, $N_i^{out}:=\{v_j|e_{ij}\in E\}$, the \textit{in-neighborhood} $N_i^{in}:=\{v_j|e_{ji}\in E\}$. The \textit{neighborhood} of $v_i$, $N_i:=N_i^{out}\cup N_i^{in}$. The \textit{degree} of vertex $v_i$, $k_i:=|N_i|$ is the number of vertices in the neighborhood of $v_i$, this can similarly be defined for in and out neigborhoods. The directional clustering coefficient for $v_i$ with $k_i>1$ can now be defined as:
$$C_i :=\frac{|\{e_{jk}\}|}{k_i(k_i-1)},$$
where $v_j,v_k\in N_i$ and $e_{jk}\in E$, let $n=|V|$ denote the number vertices the \textit{directional clustering coefficient} for $G$ is defined as:
$$C_G:=\frac{\sum_{i\in V} C_i}{n}.$$
The \textit{characteristic path length} $L_G$ of graph $G$ is the average shortest path length between vertices in $G$. Land  $L_{Gij}$ denote the length of the shortest path between vertices $i$ and $j$. 
$$L_G:= \frac{\sum_{i\in V} \sum_{j \in V\setminus i}L_{Gij}}{n(n-1)}.$$ 
Note finding all shortest paths can be performed using the Floyd-Warshall algorithm which has $O(|V|^3)$ complexity\cite{Floyd}.

For networks where $n\gg k\gg ln(n) \gg1$, the ring lattice will have $L_{lattice}\approx\frac{n}{2k}$ and $C_{lattice}\approx\frac{3}{4}$.
A large random network  $ L_{random}\approx\frac{\ln(n)}{\ln(k)}$ and $C_{random}=\frac{k}{n}$. A network is considered small world when $L_G\approx L_{random}$ and $C_G \gg C_{random}$

To find the core of most connected vertices we use the corefind algorithm, showin in Algorithm \ref{alg:findcore}.This algorithm recursively removes all vertices of degree $x$ from the network, starting at $1$. If there are vertices left in the network we repeat this with $x+1$. Once an empty network is obtained the previous network is deemed the core, because recursively removing nodes of heigher degree will inevitably lead to an empty network. The choice of the $degree(j)$ - standard, in-degree, out-degree,  or in+out-degree - facilitates in finding cores with different characteristics.

%\algsetup{indent=2em}
%\newcommand{\FindCore}{\ensuremath{\mbox{\sc FindCore}}}
%\begin{algorithm}[h!]
%\caption{$\FindCore(Network)$}\label{alg:findcore}
%\begin{algorithmic}
%\REQUIRE A Network $G=(V,E)$.
%\ENSURE The Core of Network $G$ and the degree $x$ at which the core is found.
%\medskip
%\STATE $x\gets 1$
%\WHILE {$|V| > 0$}
%	
%	\STATE $Gprevious \gets G$
%	\WHILE {$|\{j\in V | degree(j)<x\}| > 0$}
%		\STATE $G\gets removeNodes(G, \{j\in V | degree(j)<x\})$ 
%	\ENDWHILE
%	\STATE $i \gets x+1$
%\ENDWHILE
%\RETURN $Gprevious, x-1$
%\end{algorithmic}
%\end{algorithm}


\subsection{Feature extraction}
When working with images, it is usually not possible to work with the raw image data (the pixel values). The reason for this is the high dimensionality of images, which can easily exist in a space of more than a million dimensions. By extracting features from images, they can be represented in a lower dimensional feature-space.  This feature extraction process has several advantages:
\begin{itemize}
\item The data becomes computationally easier to work with due to the smaller number of dimensions
\item By using the right features, the data becomes more suitable for generalization across images
\item Reducing the dimensionality makes it easier to visualize sets of images
\item Features can have an intuitive basis, which makes it easier for non-computer-scientists to analyze (sets of) images
\end{itemize}

% Computer vision is an important and maturing engineering science. It underpins an increasing variety of applications that require the acquisition, analysis, and interpretation of visual information.
In the extraction of image features, a distinction was made between low-level statistical features and higher level cognitive-based features....

\subsubsection{Statistical features}
As statistical features, many relatively simple low-level features were extracted from the images.
The first type of statistical features that were used are color-based features, which should capture the color-usage in the artwork. Many artists produce collections of art pieces with similar colors, and should therefore be (partially) distinguishable with color-based features. For each of the three RGB channels, an average and median is calculated over all the channel values. Let $\{\mathbf{x}_{m,i,c} \}_{i=1\dots n}$ be the pixel values for image $m$ in color channel $c \in \{R,G,B \}$. The average in channel $c$ of image $m$ is then given by 

\begin{equation}
\label{avgChannel}
\mu_c(\mathbf{x}_{m}) = \frac{1}{n}\sum_{i=1}^{n} \mathbf{x}_{m,i,c} 
\end{equation}
The median in channel $c$ is given by 
\begin{equation}
\label{medChannel}
\tilde{\mathbf{x}}_{m,c} = \mathbf{x'}_{m,k,c}
\end{equation}
where $\{\mathbf{x'}_{m,i,c}\}_{i = 1\dots n}$ are the sorted pixel values of channel $c$ and $k = \mbox{round}(n/2)$.
The image is also converted into the HSV color space, from which the average and median is extracted for each channel as defined in equations \ref{avgChannel} and \ref{medChannel}. The Hue channel is given by: 

$H_{m,i} = \left\{ 
\begin{array}{ll}
0 & \mbox{if $C_{m,i} = 0$};\\
60 \left(\frac{G_{m,i}-B_{m,i}}{C_{m,i}} \mbox{mod} 6 \right) & \mbox{if $M_{m,i} = R_{m,i}$};\\
60 \left(\frac{B_{m,i}-R_{m,i}}{C_{m,i}} + 2 \right) & \mbox{if $M_{m,i} = G_{m,i}$};\\
60 \left(\frac{R_{m,i}-G_{m,i}}{C_{m,i}} + 4 \right) & \mbox{if $M_{m,i} = B_{m,i}$}; \\
\end{array} 
\right\}$
Where $M_{m,i} = \max(R_{m,i},G_{m,i},B_{m,i})$ and $C_{m,i} =  M - \min(R_{m,i},G_{m,i},B_{m,i})$. The value channel is given by $V_{m,i} =  M_{m,i}$ and the saturation channel is  by $S_{m,i} = \frac{C_{m,i}}{V_{m,i}}$

The second group of features is the edge to pixel and corner to pixel ratio. Let $\{\mathbf{x}_{m,i} \}_{i=1\dots n}$ be the pixel values of the binary edge-image produced by applying a Canny edge detector[REF] on image $m$. The edge to pixel ratio of image $m$ is then computed as $f_{e,m} = \frac{1}{n}\sum_{i=1}^{n} \mathbf{x}_{m,i} $. Let $\{\mathbf{y}_{m,i} \}_{i=1\dots n}$ be the pixel values in the binary corner image produced by a corner detector that are either $1$ if the pixel is a corner or $0$ otherwise. The corner to pixel ratio of image $m$ is then computed as  $f_{c,m} = \frac{1}{n}\sum_{i=1}^{n} \mathbf{y_{m,i}} $. These two features should be helpful in distinguishing many photography artworks from other genres such as cartoons and manga. The latter two tend to have large patches of plain color patches, which will decrease the amount of edges and corners. They are also somewhat indicative to the type of scenes in photography. A blue sky will not produce many edges or corners, whereas a busy street will.  

%$v : R^2 \rightarrow \{0,1\}$
%calculated by performing Canny edge detection on the image to construct an image of edges. The number of edge pixels in this image, divided by the total number of pixels is then used as a feature. The same is done using a corner detector. 

For the final group of features, the artworks are converted from RGB image $m$ to a greyscale intensity image $I_m$ by taking for each pixel $i$, a weighted sum of the R,G and B channels: $I_{m,i} = 0.2989R_{m,i} + 0.5870G_{m,i} + 0.1140B_{m,i} $. Let $\{\mathbf{z}_{m,i}\}_{i=1\dots n}$ be the pixel values of the greyscale intensity image of image $m$. The average intensity feature is then calculated as $f_{\mu_{I_m}} = \frac{1}{n} \sum_{i = 1}^{n} \mathbf{z}_{m,i}$ and the median intensity as $\tilde{I}_m = \mathbf{z'}_{m,k}$, where $\{\mathbf{z'}_{m,i}\}_{i = 1\dots n}$ are the sorted pixel values and $k = \mbox{round}(n/2)$. These values give information about the lightness or darkness of artworks. The intensity variance feature is computed as $\mbox{Var}(I_m) = \frac{1}{n} \sum_{i=1}^n \mathbf{z}_{m,i}$, which reacts to the contrast between lightness and darkness in images. Finally the entropy of the intensity is calculated as follows. $H(I_m) = -\sum_{u = 1}^{j} \hat{p}_u \log_2(\hat{p}_u) $, where $\{\hat{p}_u(\mathbf{z}_m)\}_{u = 1 \dots j}$ are the histogram bins of the intensity values and are defined as $\hat{p}_u(\mathbf{z}_m) = \sum_{i=1}^n \delta[b(\mathbf{z}_{m,i}) - u] $. The function $b : R \rightarrow \{1 \dots j \}$ returns the index of the bin of the input pixel value in the intensity space and $\delta[g] = 1$ if $g = 0$, otherwise $0$. This feature somewhat characterizes the texture in an image.  

The features described above only contain global information about images. In order to capture localized information as well, several of the features described above are also extracted from different regions of the image. The regions of the image are obtained by dividing the image along both dimensions into NxM equal-sized regions. Since feature values will most likely vary from region region, these compositional features should provide valuable additional information about an image. 

\textbf{add reference to opencv somewhere}


%\subsubsection{Weibull, this subsubsubsection can be made normal text, but we can do that later}
%The contrast of natural image statistics has been shown to conform to Weibull-shaped probability distributions \cite{Weibull_physical}. Furthermore, when images do not adhere to this distribution, the images in question are mashups of multiple sub-images which themselves do conform to the Weibull distribution. In addition to this property of natural images, it has been proposed that the parameters for the Weibull distribution form a basis for the description of texture in images \cite{Weibull_6}. There is indeed evidence that the human visual system is capable of approximating the parameters of the Weibull distribution \cite{Weibull_brain}. Last the two most important parameters of the Weibull distribution, when it comes to natural images have a straight forward interpretation, the shape parameter the describes the resemblance to other probability distribution, from a power-law to the normal distribution, where the scale describes the how wide the distribution is. Therefore we included the maximum likelihood estimation of the Weibull-distribution for contrast of the image as used for \cite{Weibull_6} in our Feature extraction toolbox, unfortunately this seemed to give unstable results, therefore we later eliminated it. 

\subsubsection{Cognitively-inspired features \label{proposed-cognitive}}
One of the more recent trends in computer vision research in the pursuit of human-like capability is the coupling of cognition and vision into cognitive computer vision. The term cognitive computer vision has been introduced with the aim of achieving more robust, resilient, and adaptable computer vision systems by endowing them with a cognitive faculty: the ability to learn, adapt, weigh alternative solutions, and develop new strategies for analysis and interpretation.

Recent studies focused on computational models of focal visual attention. Attention has been seen to influences the processing of visual information even in the earliest areas of primate visual cortex. Even more, it has been discovered that the interaction of bottom-up sensory information and top-down attentional influences creates an integrated \textit{saliency map}, that can be defined as a topographic representation of relative stimulus strength and behavioral relevance across visual space \cite{Saliency_WWHW}. This map enables the visual system to integrate large amounts of information, even from outside the fovea, because it provides an efficient coding scheme for the potentially most relevant information in the sensory input. 
An important model based on this theory is the one provided by Itti, Koch and Niebur \cite{Itti_review}\cite{Itti_model}. The model tries to mimic the properties of primate early vision. Despite its simple architecture the model is capable of strong performance with complex natural scene.
The model work as follow: an input image is decomposed through several pre-attentive feature detection mechanisms which operate in parallel channels over the entire visual scene, and four conspicuity maps (color, orientation, intensity and skin) are created. After few different intermediate steps, the model finally combines the four conspicuity maps into a unique saliency map. 

Til now the saliency map has been used as information channel in scene understanding and object recognition. In this reasearch, image features have extracted from the map and then used in the classification and visualization task. Features that have been extracted from those maps are: \textit{Shannon entropy} of the five maps, \textit{Standard deviation} of the distribution of attention in the saliency map, \textit{Location} of the most salient points (defined as the centers of the most salient regions) and \textit{Skin intensity} of the skin map. Skin is not a default channel in the Itti's model, but it has been found out to be really interesting and useful in devianART to distinguish artists and artworks, where there is a huge presence of photographer that create nude art. 


%\end{document}