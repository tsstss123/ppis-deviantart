\subsection{deviantART}
\subsection{Why interesting as a data source}
\subsection{Research questions}
\subsection{Solution - framework}
\subsection{Overview of the paper}

deviantART\footnote{\url{http://www.deviantart.com}} (commonly abbreviated as \textit{dA}) is one of the largest online community showcasing various forms of user-made artwork. The website was launched in 2000 and has over 13 million registered members.
The platform allows emerging and established artists to exhibit, promote, and share their works within a peer community dedicated to the arts. 
The site features a wide variety of creative expressions including animations, photographs, web skins, films, and literature, which are categorised within the customisable dA gallery according to a comprehensive structure. 
It is an highly interactive and dynamic community where members have their own profile with their galleries and images and they can explore other users profiles, comment other members artworks and follow them. Moreover, members are able to form communities of interest, link other artworks, and take inspirations.
The dataset deviantART proposes is very rich and full of interesting information. Therefor the research proposed in this paper focus on the analysis of some aspects of this community. 

%It provides users, also called deviants, to share galleries and images.  When someone visits such a gallery a featured page will be shown. Furthermore the user is provided with options to browse the gallery or visit a sub gallery defined by the user. There are various types of members: normal members, premium members, banned members, staff members, etc. Premium members have extra benefits like no ads, gallery customization, beta-test new site features and more.
% Newly added art is submitted to the constantly-changing \textit{Newest} listing, where it is viewable by the general public. 
% Members are able to create profiles, galleries of their own work, and to choose \textit{Favorites} from among other submissions.
% Through the publicity process, some pieces become \textit{Popular}, and are added to a ranked list of \textit{Most Popular} in the last 8 hours, 1 day, 3 days, 1 week, 1 month, and \textit{All time}.
% During their browsing, members are also allowed to comment on one anothers art and on their profiles, making for a highly interactive and dynamic community.

[social aspects/networks] Bart B.

[why relevant, problem with current research] Research questions\\

\begin{itemize}
\item Can we visualize important aspects of deviantART?
\item Can artists and/or styles be distinguished?
\item Are artists influencing each other?
\item Do art styles change over time?
\item Are there none-artists interesting for deviantART?
\end{itemize}

[goals - framework]