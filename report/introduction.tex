deviantART\footnote{\url{http://www.deviantart.com}} (commonly abbreviated as dA) is one of the largest online communities showcasing various forms of user-made artwork.
The website was launched in 2000 and has over 13 million registered members.
The platform allows emerging and established artists to exhibit, promote, and share their works within a peer community dedicated to the arts. 
All artwork is organized according to a comprehensive category structure, including photographs, drawings, manga and short animations.
dA is a highly interactive and dynamic community where artists have their own profile containing their artwork.
Artists can explore the profiles of other artists and leave comments on their artwork.
Each artist can add other artists to the "friends list" to automatically receive updates (e.g. newly added artwork) from these artists.
The dataset deviantART proposes is very rich and full of interesting information. There has been research into social networks, feature extraction and classification of images and image visualisation of large datasets, however there has been no research combining all these aspects. Also dA has been researched little compared to it's peers such as flickr.

This study is meant as \textbf{explorative research} on the dA community, trying to provide answers on art-related questions.
As a starting point for our explorative research we envisioned the following use case for a humanities researcher wanting to explore dA. In particular, a database of the dA network is explored with a free available complex network analysis tools such as Pajek. This enables the researcher to identify interesting artwork collections based on their role in the network. These collections can consist of groups of individual artworks, artists' galleries or categories of artworks.  All data relevant for the identified collections is retrieved and stored offline. The software automatically extracts image features from the artworks which are also stored offline. Through a simple interface a classifier is started to try to learn the boundary between classes indicated by the researcher. The result of this classifier will enable the researcher to see which features are most discriminative, but will also enable other artworks to be identified as being a member of or closely resembling one of the indicated classes. A visualiser should make this whole experience visual and intuitive, showing the art collections and how they relate to the features and classifier performace. Moreover, it relates the features, classifier performance and art collections back to the network.

Besides the use case of a humanities researcher doing research, we proposed multiple art-related questions that would be interesting for such a researcher to see answered.
\begin{itemize}
\item Can we visualize important aspects of deviantART?
\item Can artists and/or styles be distinguished?
\item Are artists influencing each other?
\item Do art styles change over time?
\item Are there none-artists interesting for deviantART?
\end{itemize}

To take steps towards making part of the use case reality as well as answering parts of the posed questions, we have embarked on three lines of research. The first was to create a database of a significat part of the network and research the network retrieved. Another line researched feature extraction and classification of about 5000 images from deviant art relating to around 30 artists. The last pillar of research is the visualiser which enables a quick visual overview of the features and artworks for the whole dataset used for feature extraction and classification.

The three lines of research as described above have resulted in a toolkit for dA research, with varying degrees of integration. This toolkit has been used to perform experiments which have resulted in identifying cores of interesting artists within the dA network. Other experiments have been based on extracted image features, to classify images based on per user classes.

This report is organised as followes, after this introduction we will describe the three pillars of research into networks, image features and classifiers and last the visualisation of large datasets. The next section is devoted to the implementation of various parts of the toolbox, as well as their level of integration. After this section we will describe the experiments which were conducted with their results. We will finish with a section on futurework after which the conclusion will conclude this report. 

%A large part of this research is devoted to finding suitable representations for the information retrieved from deviantART (especially images and artists). To do it we choose to implement a new toolkit, that not only try to visualize all the information, but also allow to compare images, artists and styles in an interactive way.

%It provides users, also called deviants, to share galleries and images.  When someone visits such a gallery a featured page will be shown. Furthermore the user is provided with options to browse the gallery or visit a sub gallery defined by the user. There are various types of members: normal members, premium members, banned members, staff members, etc. Premium members have extra benefits like no ads, gallery customization, beta-test new site features and more.
% Newly added art is submitted to the constantly-changing \textit{Newest} listing, where it is viewable by the general public. 
% Members are able to create profiles, galleries of their own work, and to choose \textit{Favorites} from among other submissions.
% Through the publicity process, some pieces become \textit{Popular}, and are added to a ranked list of \textit{Most Popular} in the last 8 hours, 1 day, 3 days, 1 week, 1 month, and \textit{All time}.
% During their browsing, members are also allowed to comment on one anothers art and on their profiles, making for a highly interactive and dynamic community.

%[social aspects/networks] Bart B.





%In order to explore dA and answer the research questions we have developed a toolkit.
%This toolkit provides functionality to capture a dataset from the dA website and annotate the images with state of the art %image features.
%These images features are used to cluster images using different classifiers.
%The visualization that is part of the toolkit is used to explore and analyze the dataset.