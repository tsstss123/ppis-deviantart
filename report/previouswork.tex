\subsection{Online social network (art - pictures)}
small world network
Eventhough dA presents a rich dataset interesting for the art, social web, and computer vision communities, research on this a dA datset is low. In \cite{DaMasters} dA discussed in the context of evaluating a specific peer-review and critiquing application.
Flickr is a large online community based around pictures, it has some resemblance with the functionality of dA provides, though it is more tailored to photographers instead of visual artists. Favorite markings and their spread through Flickr has been described in \cite{cha2009measurement}. The structure of Flickr has been discussed from an evolutionary perspective, using time information to show how the network grows, from a macroscopic \cite{kumar2006structure} and microscopic perspective \cite{leskovec2008microscopic}.

\subsection{Image features}
\subsubsection{Cognitively-inspired features}
One of the more recent trends in computer vision research in the pursuit of human-like capability is the coupling of cognition and vision into cognitive computer vision. The term cognitive computer vision has been introduced with the aim of achieving more robust, resilient, and adaptable computer vision systems by endowing them with a cognitive faculty: the ability to learn, adapt, weigh alternative solutions, and develop new strategies for analysis and interpretation.

Recent studies focused on computational models of focal visual attention. Attention has been seen to influences the processing of visual information even in the earliest areas of primate visual cortex. Even more, it has been discovered that the interaction of bottom-up sensory information and top-down attentional influences creates an integrated \textit{saliency map}, that can be defined as a topographic representation of relative stimulus strength and behavioral relevance across visual space \cite{Saliency_WWHW}. This map enables the visual system to integrate large amounts of information, even from outside the fovea, because it provides an efficient coding scheme for the potentially most relevant information in the sensory input.
An important model based on this theory is the one provided by Itti, Koch and Niebur \cite{Itti_review}\cite{Itti_model}. The model tries to mimic the properties of primate early vision. Despite its simple architecture the model is capable of strong performance with complex natural scene.

\subsection{Classification of image features}

\subsection{Visualization of image features}