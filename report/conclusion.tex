This research started with an inspirational \textit{usecase} and interesting research questions in humanity.
Some aspects of the \textit{usecase} have been implemented in the toolkit and some experiments have been conducted, both aimed to answer some of the research questions.
This has been the first research into deviantART as well as the first attempt to combine online social networks with image analysis.

The toolkit provides functionality to capture a rich dataset containing images from the deviantART website, after which the images are annotated with different types of image features. These image features are used to classify images using different classifiers. The visualization is used to explore and analyze the dataset.

The research has shown that users can be identified based on the location of their artwork in feature space. Furthermore, some image features have been shown to separate artist collections and styles. The best classification result was achieved using a linear SVM classifier.

Moreover, cores of interesting deviantART artists have been identified by analyzing the network. These cores exhibit small-world topology. 
