deviantART\footnote{\url{http://www.deviantart.com}} (commonly abbreviated as dA) is one of the largest online communities showcasing various forms of user-made artwork.
The website was launched in 2000 and has over 13 million registered members.
The platform allows emerging and established artists to exhibit, promote, and share their works within a peer community dedicated to the arts. 
All artwork is organized according to a comprehensive category structure, including photographs, drawings, manga and short animations.
dA is a highly interactive and dynamic community where artists have their own profile containing their artwork.
Artists can explore the profiles of other artists and leave comments on their artwork.
Each artist can add other artists to the \textit{friends list} to automatically receive updates (e.g. newly added artwork) about these artists.
The dataset deviantART proposes is very rich and full of interesting information. 
There has been research into social networks, feature extraction and classification of images and image visualization of large datasets, however there has been no research combining all these aspects. Even more, dA has been researched little in contrast to it's peers such as \textit{Flickr}.

This study is meant as \textit{exploratory research} on the dA community, trying to provide answers on art-related questions.
As a starting point for our exploratory research we envisioned the following use case for a humanities researcher wanting to explore dA. In particular, a database of the dA network is explored with a free available complex network analysis tools such as Pajek\footnote{\url{http://vlado.fmf.uni-lj.si/pub/networks/pajek/}}. This enables the researcher to identify interesting artwork collections based on their role in the network. These collections can consist of groups of individual artworks, artists' galleries or categories of artworks.  All data relevant for the identified collections is retrieved and stored offline, such data includes, the digital artworks themselves, thumbnails representing smaller versions of the artwork, the artist who created the artwork, the date of submission to dA, the category to which the artwork is assigned. The software automatically extracts image features from the artworks which are also stored offline. Through a simple interface a classifier is started to try to learn the boundary between classes indicated by the researcher. The result of this classifier will enable the researcher to see which features are most discriminative, but will also enable other artworks to be identified as being a member of or closely resembling one of the indicated classes. A visualization application should make this whole experience visual and intuitive, showing the art collections and how they relate to the features and classifier performance. Moreover, it relates the features, classifier performance and art collections back to the network.

Besides the use case of a humanities researcher doing research, we proposed multiple art-related questions that would be interesting for such a researcher to see answered.
\begin{itemize}
\item Can we visualize important aspects of deviantART?
\item Can artists and/or styles be distinguished?
\item Are artists influencing each other?
\item Do art styles change over time?
\item Are there none-artists (dA users that do not produce art, but favorite art from other artists) that are interesting for deviantART?
\end{itemize}

To take steps towards making part of the use case reality as well as answering parts of the posed questions, we have embarked on three lines of research. The first was to create a database of a significant part of the network and research the network retrieved. Another line was researching on the extraction of visual feature and classification of about 5000 images from deviantART relating to around 30 artists. The last pillar of research is the visualization application which enables a visual overview of the features and artworks for the whole dataset used for feature extraction and classification.

The three lines of research as described above have resulted in a toolkit for dA research, with varying degrees of integration. This toolkit has been used to perform experiments which have resulted in identifying cores of interesting artists within the dA network. Other experiments have been based on extracted image features, to classify images based on per user classes.

This report is organized as follows, after this introduction in section \ref{intro}, we describe the three pillars of research into networks (\ref{net}), image features and classifiers (\ref{feat_class}) and last the visualization of large datasets (\ref{visualiz}). Section \ref{toolkit} is devoted to the implementation of various parts of the toolkit, as well as their level of integration. In section \ref{exp} we describe the experiments which have been conducted and their results. Finally, in section \ref{future_work} we discuss the future work and in section \ref{conclusion} we draw a conclusion about the researches done in this paper and their results. 