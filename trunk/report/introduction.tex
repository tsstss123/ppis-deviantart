\textbf{[about deviantART] Nick}
\\
deviantART\footnote{\url{http://www.deviantart.com}} (commonly abbreviated as \textit{dA}) is the largest online community [?] showcasing various forms of user-made artwork.
The website was launched in 2000 and has over 13 million registered members.
The platform allows emerging and established artists to exhibit, promote, and share their works within a peer community dedicated to the arts.
Newly added art is submitted to the constantly-changing \textit{Newest} listing, where it is viewable by the general public. 
Members are able to create profiles, galleries of their own work, and to choose \textit{Favorites} from among other submissions.
Through the publicity process, some pieces become \textit{Popular}, and are added to a ranked list of \textit{Most Popular} in the last 8 hours, 1 day, 3 days, 1 week, 1 month, and \textit{All time}.
During their browsing, members are also allowed to comment on one another�s art and on their profiles, making for a highly interactive and dynamic community.
All artwork is organized in a comprehensive category structure, including traditional media, such as painting and sculpture, to digital art, pixel art, films and anime.

[social aspects/networks] Bart B.

[why relevant, problem with current research] Research questions\\
\begin{itemize}
\item Can we visualize important aspects of deviantART?
\item Can artists and/or styles be distinguished?
\item Are artists influencing each other?
\item Do art styles change over time?
\item Are there none-artists interesting for deviantART?
\end{itemize}

[goals - framework]