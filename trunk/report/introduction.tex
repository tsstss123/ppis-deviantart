[about deviantART] Nick

\textbf{[about image features] Bart P.}
\\
When working with images, it is usually not possible to work with the raw image data (the pixel values). The reason for this is the high dimensionality of images, which can easily exist in a space of more than a million dimensions. By extracting features from images, they can be represented in a lower dimensional feature-space.  This feature extraction process has several advantages:
\begin{itemize}
\item The data becomes computationally easier to work with due to the smaller number of dimensions
\item By using the right features, the data becomes more suitable for generalization across images
\item Reducing the dimensionality makes it easier to visualize sets of images
\item Features can have an intuitive basis, which makes it easier for non-computer-scientists to analyze (sets of) images
\end{itemize}
\textit{Here something general about different kinds of image features}

[social aspects/networks] Bart B.

[why relevant, problem with current research] Quang

[goals] Bart B.