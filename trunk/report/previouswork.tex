\subsection{Online social network (art - pictures)}
small world network
Eventhough dA presents a rich dataset interesting for the art, social web, and computer vision communities, research on this a dA dataset is low. In \cite{DaMasters} dA discussed in the context of evaluating a specific peer-review and critiquing application.
Flickr is a large online community based around pictures, it has some resemblance with the functionality of dA provides, though it is more tailored to photographers instead of visual artists. Favorite markings and their spread through Flickr has been described in \cite{cha2009measurement}. The structure of Flickr has been discussed from an evolutionary perspective, using time information to show how the network grows, from a macroscopic \cite{kumar2006structure} and microscopic perspective \cite{leskovec2008microscopic}.

\subsection{Image features}
\subsubsection{Cognitively-inspired features}
One of the more recent trends in computer vision research in the pursuit of human-like capability is the coupling of cognition and vision into cognitive computer vision. The term cognitive computer vision has been introduced with the aim of achieving more robust, resilient, and adaptable computer vision systems by endowing them with a cognitive faculty: the ability to learn, adapt, weigh alternative solutions, and develop new strategies for analysis and interpretation.

Recent studies focused on computational models of focal visual attention. Attention has been seen to influences the processing of visual information even in the earliest areas of primate visual cortex. Even more, it has been discovered that the interaction of bottom-up sensory information and top-down attentional influences creates an integrated \textit{saliency map}, that can be defined as a topographic representation of relative stimulus strength and behavioral relevance across visual space \cite{Saliency_WWHW}. This map enables the visual system to integrate large amounts of information, even from outside the fovea, because it provides an efficient coding scheme for the potentially most relevant information in the sensory input.
An important model based on this theory is the one provided by Itti, Koch and Niebur \cite{Itti_review}\cite{Itti_model}. The model tries to mimic the properties of primate early vision. Despite its simple architecture the model is capable of strong performance with complex natural scene.

\subsection{Classification of image features}

\subsection{Visualization of image features}
Image features and information visualization are both research fields that have received great interest.
Research that combines both fields is largely coming from the image retrieval field, focussing on efficient methods to retrieve images from (large) image collections.

Musha et al~\cite{musha1998interface} developed a visualization method and an interface for image retrieval.
In their method, principal component analysis (PCA) is dynamically applied to the to image features of the retrieved images in order to determine their eigenspace and the retrieved images are displayed in that space.
Statistical experiments showed that heir method effectively decentralizes the retrieved images over the two-dimensional space.

Chen et al.~\cite{chen2000content} compared and analyzed a number of Pathfinder networks of images generated based on low-Ievel image features (color, texture and shape).
Salient structures of images are visualized according to features extracted .from color, texture, and shape orientation.

Schneidewind et al~\cite{schneidewind2004approach} presented a visualization technique that aims to provide a tool to analyze a mismatch between the user�s perception and the system�s calculation of similarity.
They combine techniques of visual image retrieval and information visualization to acquire insight into the extracted feature data.
They implemented three visualization techniques to present feature data on three different levels of abstraction: Data Table, a Parallel Coordinate Plot, and a Color Space Plot.

Imo et al~\cite{imo2008interactive} propose to make content based image retrieval systems more �transparent� by visualization of the employed features.
Since non-experts should be able to operate the CBIR system, they argue that features should be visualized as prototypical, artificial images, rather than feature-specific visualizations.
% sidetrack?

Yang el al~\cite{yang2006semantic} propose a scalable semantic image browser by applying existing information visualization techniques to semantic image analysis.
The system consists of multiple visualization components that allows effective high dimensional visualization without dimension reduction.
The Multi-Dimensional Scaling image view maps image miniatures onto the screen based on their content similarities using a fast MDS algorithm.
Multiple interactions are provided to reduce clutter (reordering, dynamic scaling, relocation, distortion, showing original image, zooming and panning).
The Value and Relation content view visually represents the contents of the whole image collection.
The correlations among different contents within the image collection, as well as detailed annotations of each image, are visually revealed in this view.
The concept-sensitive image content analysis technique was used as automatic annotation engine.
This technique abstracts image contents by automatically detecting the underlying salient objects in images and associating them with the corresponding semantic objects and concepts according to their perceptual properties.