% Here the formulas and description of what we used theoretically
%\documentclass{article}
%\begin{document}
\subsection{Networks}

Three network models of particular relevance for this project are the small-world model \cite{watts1998collective} from Watts and Strogatz and the Erd\"{o}s-R\'{e}nyi model for random networks \cite{erdos1960evolution} and regular ring latices. Small-world networks inhabit the space between the regular networks or lattices and random networks. Many practical networks have been shown to have a small-world topology, such as the nternet, the power grid amd neural networks,  for some more examples see \cite{albert2002statistical}. Small world-networks have a high cluster coefficient and low characteristic path length, means that  whereas random networks have a low characteristic path and low cluster coefficient and lattices have high cluster coefficients but long characteristic path length.

Coloquially explained in terms of my social network, a large(close to 1 ) cluster coefficient means that many of my friends know each other. A low characteristic path length means that I am connected to anyone on the world through small friends-of-friends chains.

Formally, let $G=(V, E)$ be a graph, where $V$ is a set of vertices, and $E$ a set of edges between vertices in $V$, here $e_{ij}$ denotes the edge connecting $i$ and $j$.  The \textit{neighborhood} of $v_i$, $$N_i:=\{v_j|e_{ij}\in E \lor e_{ji}\in E\}.$$ The \textit{degree} of vertex $v_i$, $k_i:=|N_i|$ is the number of vertices in the neighborhood of $v_i$. The directional clustering coefficient for $v_i$can now be defined as:
$$C_i :=\frac{|\{e_{jk}\}|}{k_i(k_i-1)},$$
where $v_j,v_k\in N_i$ and $e_{jk}\in E$, let $n=|V|$ denote the number vertices the \textit{directional clustering coefficient} for $G$ is defined as:
$$C_G:=\frac{\sum_{i\in V} C_i}{n}.$$
The \textit{characteristic path length} $L_G$ of graph $G$ is the average shortest path length between vertices in $G$. Land  $L_{Gij}$ denote the length of the shortest path between vertices $i$ and $j$. 
$$L_G:= \frac{\sum_{i\in V} \sum_{j \in V\setminus i}L_{Gij}}{n(n-1)}.$$ 
Note finding all shortest paths can be performed using the Floyd-Warshall algorithm which has $O(|V|^3)$ complexity\cite{Floyd}.

For networks where $n\gg k\gg ln(n) \gg1$, the ring lattice will have $L_{lattice}\approx\frac{n}{2k}$ and $C_{lattice}\approx\frac{3}{4}$.
A large random network  $ L_{random}\approx\frac{\ln(n)}{\ln(k)}$ and $C_{random}=\frac{k}{n}$. A network is considered small world when $L_G\approx L_{random}$ and $C_G \gg C_{random}$

To find the core of most connected components we recursively remove all vertices from our network of degree $x$, starting at $1$. If there are vertices left in the network we repeat this with $x+1$. Once an empty network is obtained the previous network is deemed the core, because  recursively removing nodes of heigher degree will inevitably lead to an empty network. 

\algsetup{indent=2em}
\newcommand{\FindCore}{\ensuremath{\mbox{\sc FindCore}}}
\begin{algorithm}[h!]
\caption{$\FindCore(Network)$}\label{alg:findcore}
\begin{algorithmic}
\REQUIRE A Network $G=(V,E)$.
\ENSURE The Core of Graph $G$ and the degree $x$ at which the core is found.
\medskip
\STATE $x\gets 1$
\WHILE {$|V| > 0$}
	\STATE $remove\gets \{j\in V | k_j<x\}$
	\STATE $Gprevious \gets G$
	\WHILE {$|remove| > 0$}
		\STATE $G\gets removeNodes(G, remove)$ 
	\ENDWHILE
	\STATE $i \gets x+1$
\ENDWHILE
\RETURN $Gprevious, x-1$
\end{algorithmic}
\end{algorithm}

%\end{document}